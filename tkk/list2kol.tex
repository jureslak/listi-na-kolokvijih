\documentclass[a4paper, oneside, 10pt]{article}
\usepackage[slovene]{babel}
\usepackage[utf8]{inputenc}
\usepackage[T1]{fontenc}
\usepackage{url}
\usepackage{graphicx}
\usepackage[usenames]{color}
\usepackage[reqno]{amsmath}
\usepackage{amssymb, amsthm}
\usepackage{enumerate}
\usepackage{array}
\usepackage[bookmarks, colorlinks=true, %
linkcolor=black, anchorcolor=black, citecolor=black, filecolor=black, %
menucolor=black, runcolor=black, urlcolor=black%
]{hyperref}
\usepackage[
    paper=a4paper,
    top=1.5cm,
    bottom=1.5cm,
%    textheight=24cm,
    textwidth=17cm,
    ]{geometry}

\usepackage{icomma}
\usepackage{units}

\newtheorem{izrek}{Izrek}
\newtheorem{posledica}{Posledica}

\theoremstyle{definition}
\newtheorem{definicija}{Definicija}
\newtheorem{opomba}{Opomba}
\newtheorem{zgled}{Zgled}

\def\R{\mathbb{R}}
\def\N{\mathbb{N}}
\def\Z{\mathbb{Z}}
\def\C{\mathbb{C}}
\def\Q{\mathbb{Q}}

\newcommand{\mytitle}{Teorija kodiranja in kriptografija}
\title{\mytitle}
\author{Jure Slak}
\date{\today}
\hypersetup{pdftitle={\mytitle}}
\hypersetup{pdfauthor={Jure Slak}}
\hypersetup{pdfsubject={}}

\setlength{\parindent}{0pt}
\setlength{\parskip}{8pt}

\begin{document}
\pagestyle{empty}

\begin{center}
  \bf \Large Teorija kodiranja in kriptografija
\end{center}

\textbf{Osnovne definicije.} Kod $C$ nad končno abecedo $\Sigma$ je končna podmnožica
$\Sigma^\ast$. Množico $\Sigma$ imenujemo kodna abeceda, elemente $C$ kodne besede in
elemente $\Sigma$ kodni simboli. Kod $C$ je bločni, če za nek $n$ velja $C \subset
\Sigma^n$, torej vse besede so enako dolge. Razmaknjenost koda $C$ je minimalna razdalja
med dvema različnima besedama. 

$(n,M,d)$ kod je kod z $M$ besedami dolžine $n$ in razmaknjenostjo $d$. Kod z
razmaknjenostjo $d$ odkrije $d-1$ napak in popravi $\lfloor\frac{d-1}{2}\rfloor$ napak.

\textbf{Dvojiški simetrični kanal.} Pri pošiljanju se dolžina binarne besede ne spremeni,
na vsakem mestu pa se bit pokvari z enako verjetnostjo $p < 1/2$.

\textbf{Postopek dekodiranja.} Pravilo najmanjše napake. Pri prejetem $y \in \Sigma^n$ je
$x$ tista beseda iz $C$, da je $P(x|y)$ največja. Pravilo največje verjetnosti. $x$ je
tista beseda pri kateri je $(y|x)$ največja. Če so vse besede enako verjetne dajeta
pravili enak rezultat. Pravilo najbližjega soseda. $y$ dekodiramo v tisto besedo $x$ kjer
je $d(x,y)$ najmanjša. V BSC s $p<1/2$ sta pravilo največje verjetnosti in pravilo
najbližjega soseda ekvivalentni.

\textbf{Malo verjetnosti}
$P(x|y) = \frac{P(y|x)P(x)}{P(y)}$. $P(y) = \sum_{c\in C}P(y|c)P(c)$. $P(y|x) =
p^{d(x,y)}(1-p)^{n-d(x,y)}$.

\textbf{Linearni kodi.} Kod je linearen, če je vektorski podprostor $\Sigma^n$, torej zaprt
za seštevanje in množenje s skalarjem. Označimo $[n,k,d]$ kod je kod z besedami dolžine
$n$, ki je $k$ dimenzionalen vektorski prostor nad $\Sigma$ z razmaknjenostjo $d$. Če je
kod nad $GF(q)$ potem $M = q^k$. Generatorska matrika je matrika dimenzije $k\times n$, ki
ima v vrsticah linearno neodvisne kodne besede. Matrika je v standardni obliki, če je na
začetku $k\times k$ identiteta. Kodiramo tako, da besedilo dolžine $k$ pomnožimo z matriko: $s\cdot
G$. V standardni obliki se besedilo ohrani doda se le nekaj solate na konec. 

$C^\perp = \{ x\in \Sigma^n, C\cdot x = \{0\}\}$ je dualni kod koda $C$. Njegovo
generatorsko matriko označimo s $H$ in jo imenujemo nadzorna matrika, in je dimenzije $n-k\times n$. Velja $GH^\top=0$ in $x
\in C \iff Hx^\top = 0$. Če je $G$ v standardni obliki $[I_k| A]$ je $H = [-A^\top|I_{n-k}]$ ena
izmed nadzornih matrik te matrike. 

\textbf{Ekvivalentni kodi.}
Gaussove operacije po vrsticah koda ne spremenijo. Menjavanje stolpcev in množenje
stolpcev s skalarjem pa spremeni kod v ekvivalenten kod, potem lahko npr. zračunamo
nadzorno matriko, ki jo pretvorimo z enakimi operacijami nazaj v nadzorno matriko
originalnega koda.

\textbf{Dekodiranje s sindromi.} Za vse možne napake (primerne teže) izračunamo sindrome:
$He^\top$. Za dano prejeto besedo $y$ prav tako izračunamo njen sindrom $Hy^\top$ in vidimo s
katero napako se ujema (ali pa je 0, kar pomeni da smo dobili veljavno besedo, ali pa se
ne ujema z nobeno, kar pomeni da te napake ne moremo popraviti). Nato od prejetega sporočila odštejemo to napako.

Razmaknjenost linearnega koda $C$ je največji tak $d$, da je vsakih $d-1$  stolpcev
nadzorne matrike $H$ še linearno
neodvisnih. Ekvivalentno, $d$ je najmanjša teža neničelne besede v $C$.

\textbf{Meje za kode.} Naj bo $A_q(n,d) = \max\{M; \exists (n,M,d)\text{-kod nad
}GF(q)\}$. $A_q(n,1) = q^n$ $A_q(n,2) = q^{n-1}$. Kod je popoln če dosega Hammingovo mejo.\\
Hammingova meja: $A_q(n,d) \le \frac{q^n}{\sum_{k=0}^{\lfloor\frac{d-1}{2}\rfloor}\binom{n}{k}(q-1)^k}$.
Singeltonova meja: $A_q(n,d) \leq q^{n-d+1}$. \\ 
Gilbert-Varshamova meja: $A_q(n,d) \geq \frac{q^n}{\sum_{k=0}^{d-1}\binom{n}{k}(q-1)^k}$.
Linearni kod lahko popravi največ $\lfloor\frac{n-k}{2}\rfloor$ napak.

\textbf{Ciklični kodi.} $[n,k,d]$-kod je cikličen, če je linearen in vsebuje vse ciklične pomike vseh
besed. Ciklični pomik je množenje s $t$ v $(GF(q)[t])/(t^n-1)$ in vsak cikličen kod
ustreza nekemu idealu, ki je glavni $(g(t))$. Torej $g(t)$ deli $t^n-1$. Baza koda $\{g(t), tg(t), \dots,
t^{k-1}g(t)\}, k = n-\deg(g)$.

\textbf{Reed-Salomonovi kodi.} $\beta$ primitivni element $GF(2^k)$. $RS(n,k)$ je cikličen
linearen kod dolžine $n=2^r-1$ dimenzije $k = n-\delta+1$ nad $GF(2^r)$ generiran s polinomom
$g(t) = (t-\beta)(t-\beta^2)\cdots(t-\beta^{\delta-1})$. Za RS-kode velja, da je
razmaknjenost $d = n-k+1 = \delta$. RS kodi dosežejo Singeltonovo
mejo -- popravijo največ napak glede na število dodanih bitov.

\textbf{Shanonova teorija.} Nope.

\end{document}
% vim: spell spelllang=sl
% vim: foldlevel=99
