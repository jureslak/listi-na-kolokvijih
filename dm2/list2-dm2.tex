\documentclass[8pt,a4paper]{amsart}
% ukazi za delo s slovenscino -- izberi kodiranje, ki ti ustreza
\usepackage[slovene]{babel}
%\usepackage[cp1250]{inputenc}
%\usepackage[T1]{fontenc}
\usepackage[utf8]{inputenc}
\usepackage{amsmath,amssymb,amsfonts}
\usepackage{url}
%\usepackage[normalem]{ulem}
\usepackage{enumerate}
\usepackage[dvipsnames,usenames]{color}
\usepackage{bbold}



\usepackage[
top    = 1cm,
bottom = 1cm,
left   = .5cm,
right  = 0.5cm]{geometry}
%
%% ne spreminjaj podatkov, ki vplivajo na obliko strani
%\textwidth 19cm
%\textheight 27cm
%\oddsidemargin-1.5cm
%\evensidemargin-1.5cm
%\topmargin-30mm
%%\addtolength{\footskip}{10pt}
%\pagestyle{plain}
%%\overfullrule=15pt % oznaci predlogo vrstico


% ukazi za matematicna okolja
\theoremstyle{definition} % tekst napisan pokoncno
\newtheorem{definicija}{Definicija}[section]
\newtheorem{primer}[definicija]{Primer}
\newtheorem{opomba}[definicija]{Opomba}
\newtheorem{zgled}[definicija]{Zgled}

\theoremstyle{plain} % tekst napisan posevno
\newtheorem{lema}[definicija]{Lema}
\newtheorem{izrek}[definicija]{Izrek}
\newtheorem{trditev}[definicija]{Trditev}
\newtheorem{posledica}[definicija]{Posledica}


\newcommand{\R}{\mathbb R}
\newcommand{\N}{\mathbb N}
\newcommand{\Z}{\mathbb Z}
\newcommand{\C}{\mathbb C}
\newcommand{\Q}{\mathbb Q}

\begin{document}
\thispagestyle{empty}
\setlength{\parindent}{0pt}

\section{Permutacijske grupe}

\textsc{Definicija:} Naj bo $G$ množica nekaterih permutacij nad množico $X$. Če
$G$ tvori grupo za komponiranje, pravimo, da je $G$ \textbf{permutacijska
grupa}, ki deluje na $X$.

Naj grupa $G$ deluje na $X$. Definiramo relacijo: $x \sim y \Longleftrightarrow
\exists g \in G: g(x)=y$.

\textsc{Trditev:} $\sim$ je ekvivalenčna relacija na $X$.

\textsc{Definicija:} \textbf{Orbite} (glede na delovanje $G$ na $X$) so
ekvivalenčni razredi relacije $\sim$, velja torej: $Gx = \{ y \in X; g(y) = x
\}$.

$Gx$... orbita elementa $x$

$G(x \rightarrow y) = \{ g \in G; g(x)=y \}$

$G_x$... stabilizator elementa $x$: $G(x \rightarrow x)$

\textsc{Izrek:} Če je $G$ končna permutacijska grupa, ki deluje na $X$, tedaj je
za vsak $x \in X$:  $|G| = |Gx| |G_x|$.

\textsc{Definicija}: Naj bo $G$ grupa, ki deluje na $X$. Za $g \in G$ je $F(g) =
\{ x\in X; g(x) = x \}$ množica negibnih točk permutacije $g$.

\textsc{Izrek:} Število orbit pri delovanju $G$ na $X$ je enako:  $\frac{1}{|G|}
\sum_{g \in G}|F(g)|$.

\textsc{Definicija:} Naj bo $G$ grupa in $X$ množica. \textbf{Reprezentacija}
$G$ s permutacijami nad $X$ je predpis $g \in G \mapsto \hat{g} \text{
permutacija } X$, tako da je $\widehat{g_1g_2} = \widehat{g_1}\widehat{g_2}$ za
vse $g_1,g_2 \in G$.

$\widehat{G} = \{ \widehat{g} ; g\in G \}$ je (permutacijska) grupa.

\textsc{Definicija:} Reprezentacija je \textbf{zvesta}, če je $\widehat{g_1} =
\widehat{g_2} \Longleftrightarrow g_1 = g_2$.

\textsc{Trditev:} Vsaka končna grupa premore zvesto reprezentacijo.


%%%%%%%% SIMETRIJE IN ŠTETJE
\section{Simetrije in štetje}

Naj bo $\alpha_i$ število disjunktnih ciklov dolžine $i$ v $\pi$ zapisanem kot
produkt disjunktnih ciklov. ($\alpha_1 = $ število negibnih točk $\pi$.)

Če $|\pi| = n$, potem $\alpha_i + 2\alpha_2 + \ldots + n\alpha_n= n$.

$z(\pi ; x_1,\ldots ,x_n)=x_1^{\alpha_1}x_2^{\alpha_2}\cdots x_n^{\alpha_n}$
imenujemo \textbf{ciklični indeks permutacije $\pi$}

\textsc{Definicija:} $G$ permutacijska grupa, tedaj je \textbf{ciklični indeks
grupe} $G$:
$$ Z(G;x_1,\ldots ,x_n) = \frac{1}{|G|} \sum_{g\in G} z(g;x_1,\ldots ,x_n).$$

Vrtiljaku ustreza ciklična grupa, ogrlici pa diedrska. $D_{2n}$ je grupa
simetrij pravilnega $n$-kotnika.

\textsc{Izrek:} $Z(C_n; x_1,\ldots, x_n) = \frac{1}{n} \displaystyle \sum_{d
\mid n} \phi
(d)x_d^{\frac{n}{d}}$, \quad $\phi(2^n) = 2^n-1$.

\textsc{Izrek:} $Z(D_{2n}; x_1,\ldots,x_n) = \frac{1}{2} Z(C_n;x_1,\ldots,x_n) +
\begin{cases} \frac{1}{4}(x_1^2x_2^{\frac{n}{2}-1}+x_2^{\frac{n}{2}});& n \text{
  sod} \\ \frac{1}{2}x_1x_2^{\frac{n-1}{2}};& n \text{ lih} \end{cases}$

Delovanje na ploskve nekega telesa je enako kot delovanje na oglišča dualnega
telesa. Telesa in njihovi duali:
\begin{itemize}
  \item kocka $\leftrightarrow$ oktaeder
  \item tetraeder $\leftrightarrow$ tetraeder
  \item ikozaeder (12 oglišč, ploskve trikotniki) $\leftrightarrow$ dodekaeder (20
    oglišč, ploskve petkotniki)
\end{itemize}

\begin{tabular}{| c || c | c | c |}\hline
  polieder & $|X|$ & $|G|$ & $Z$ \\ \hline\hline
  tetraeder & 4 & 12 & $\frac{1}{12}(x_1^4+8x_1x_3 + 3x_2^2$\\ \hline
  oktaeder & 6 & 24 & $\frac{1}{24}(x_1^6+6x_1^2x_4 + 3x_1^2x_2^2+6x_2^3+8x_3^2$\\ \hline
  kocka & 8 & 24 & $\frac{1}{24}(x_1^8+8x_1^2x_2^2 + 9x_2^4 + 6x_4^2)$ \\ \hline
  ikozaeder & 12 & 60 & $\frac{1}{60}(x_1^{12} + 24x_1^2x_5^2 + 15x_2^6 + 20x_3^4)$ \\ \hline
  dodekaeder & 20 & 60 & $\frac{1}{60}(x_1^{20} + 20x_1^2x_3^6 + 15x_2^{10} + 24x_5^4)$ \\ \hline
\end{tabular}

\section{Število neekvivalentnih barvanj}

$G$ grupa, ki deluje na $X$, $|X|=n$, $K$ naj bo množica $r$-barv, $w: X
\longrightarrow K$ je $r$-barvanje $X$, $\Omega = \{ w: X \longrightarrow K \}$,
$|\Omega| = r^n$

$\widehat{g}: \Omega \longrightarrow \Omega$ ($w \mapsto \widehat{g}(w)$). $g$
je avtomorfizem grafa. Velja: $(\widehat{g}(w))(x) = w(g^{-1}(x))$.

\textsc{Lema:} Preslikava~~$\widehat{\cdot}$~~je zvesta reprezentacija grupe
$G$.

Grupi $G$ in $\widehat{G} = \{ \widehat{g}: \Omega \longrightarrow \Omega \}$
sta izomorfni.

\textsc{Definicija:} Barvanji sta \textbf{ekvivalentni}, če sta v isti orbiti
grupe $\widehat{G}$, oz. število neekvivalentnih barvanj $X$ glede na $G$ je
število orbit $G$.

\textsc{Izrek:} Naj bo $G$ grupa, ki deluje na $X$ in $r\geq 2$. Tedaj je
število neekvivalentnih barvanj $X$ enako $Z(G;r,\ldots ,r)$.

$K = \{ a,b,\ldots k\}$, $U(a,b,\ldots, k)$... rodovna funkcija za vsa
neekvivalentna barvanja glede na delovanje grupe $G$ na $n$-množico $X$.

\textsc{Izrek Polya:} Če $G$ deluje na $n$-množico $X$ in je $K = \{ a,b,\ldots
,k\}$ množica barv, tedaj je $$ U(a,b,\ldots, k) = Z(G;\sigma_1,\ldots
,\sigma_n),\text{ kjer je } \sigma_i = a^i + b^i + \cdots + k^i \quad (1\leq i
\leq n) $$

\section{Ramseyeva teorija}

\textsc{Trditev:} Naj bodo povezave $K_n$ pobarvane z dvema barvama in naj bo
$r_i$ število povezav iz $i$-tega vozlišča barve 1. Tedaj je število
monokromatičnih trikotnikov enako $\binom{n}{3} - \frac{1}{2}\sum_{i=1}^n
r_i(n-1-r_i)$.

\textsc{Posledica:} V situaciji iz zadnje trditve imamo vsaj $\binom{n}{3} -
\lfloor \frac{n}{2} \lfloor (\frac{n-1}{2})^2 \rfloor \rfloor $ monokromatičnih
trikotnikov.

\textsc{Ramseyev izrek:} Naj bo $r \geq 1$ in $a_1, a_2 \geq r$. Tedaj obstaja
tako najmanjše naravno število $N(a_1,a_2;r)$, da velja naslednje: naj bo $S$
$n$-množica, kjer je $n \geq N(a_1,a_2;r)$ in recimo, da smo vse njene
$r$-podmnožice pobarvali z barvo 1 oz. barvo 2. Tedaj $S$ premore
$a_1$-podmnožico, tako da so vse njene $r$-podmnožice barve 1, ali pa $S$
premore $a_2$-podmnožico, da so vse njene $r$-podmnožice barve 2.

\textsc{Posledica:} $N(a_1,a_2;r)\leq N(N(a_1-1,a_2;r),N(a_1,a_2-1;r);r-1)+1$.

\textsc{Izrek:} $N(a_1,a_2;2) \leq \binom{a_1+a_2-2}{a_1-1}$.

$r=2$:
\begin{tabular}{|c||c|c|c|c|c|c|c|c|} \hline
$a_1 \backslash a_2$ & 3 & 4 & 5 & 6 & 7 & 8 & 9 & 10 \\ \hline \hline
3 & 6 & 9 & 14 & 18 & 23 & 28 & 36 & 40/42 \\ \hline
4 &   & 18 & 25 & 36/41 & 49/61 & 58/84 & 73/115 & 92/149 \\ \hline
5 & & & 43/49 & 58/87 & 80/143 & 101/216 & 126/316 & 144/442 \\ \hline
6 & & & & 102/165 & 113/298 & 132/495 & 169/780 & 179/1171 \\ \hline

\end{tabular}

\textsc{Izrek:} Če je $a \geq 3$, tedaj je $N(a,a;2) \geq 2^{\frac{a}{2}}$.

\textsc{Izrek (Erdős, Szekeres):} Za vsak $n \geq 3$ obstaja tako najmanjše
naravno število $N$, tako da če imamo $N$ točk v ravnini v splošni legi (nobene
3 niso kolinearne), potem med njimi obstaja $n$ točk, ki določajo konveksen
$n$-kotnik.

\textsc{Definicija:} Naj bodo $G_1,\ldots ,G_k$ grafi. \textbf{Grafovsko
Ramseyevo število} $N(G_1,\ldots ,G_k)$ je najmanjši tak $N$, da če povezave
polnega grafa $K_N$ pobarvamo poljubno z barvami $1, 2, \ldots ,k$, tedaj v tem
$K_N$ najdemo vsaj en $G_i$, ki je barve $i$.

\textsc{Izrek:} Če je $T$ drevo z $n$ vozlišči, tedaj je $N(T,K_n) = (n-1)(n-1)+1$.


\hfill Avtor: Klemen Sajovec, manjši popravki: Jure Slak

\end{document}
