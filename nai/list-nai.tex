\documentclass[a4paper,12pt]{article}
\usepackage[slovene]{babel}
\usepackage[utf8]{inputenc}
\usepackage[T1]{fontenc}
\usepackage{lmodern}
\usepackage{url}
\usepackage{graphicx}
\usepackage[usenames]{color}
\usepackage[reqno]{amsmath}
\usepackage{amssymb,amsthm}
\usepackage{enumerate}
\usepackage{array}
\usepackage[bookmarks, colorlinks=true, %
linkcolor=black, anchorcolor=black, citecolor=black, filecolor=black,%
menucolor=black, runcolor=black, urlcolor=black, pdfencoding=unicode%
]{hyperref}
\usepackage[
  paper=a4paper,
  top=1.5cm,
  bottom=1.5cm,
%    textheight=24cm,
  textwidth=18cm,
  ]{geometry}

\usepackage{icomma}
\usepackage{units}

\newtheorem{izrek}{Izrek}
\newtheorem{posledica}{Posledica}

\theoremstyle{definition}
\newtheorem{definicija}{Definicija}
\newtheorem{opomba}{Opomba}
\newtheorem{zgled}{Zgled}

\def\R{\mathbb{R}}
\def\N{\mathbb{N}}
\def\Z{\mathbb{Z}}
\def\C{\mathbb{C}}
\def\Q{\mathbb{Q}}
\def\P{\mathbb{P}}

\newenvironment{itemize*}%
{
\vspace{-6pt}
\begin{itemize}
\setlength{\itemsep}{0pt}
\setlength{\parskip}{2pt}
}
{\end{itemize}}

\newenvironment{enumerate*}%
{
\vspace{-6pt}
\begin{enumerate}
\setlength{\itemsep}{0pt}
\setlength{\parskip}{2pt}
}
{\end{enumerate}}

\title{NAI}
\author{Vesna Iršič, Jure Slak}
\date{February 2016}

\pagestyle{empty}
\setlength{\parindent}{0pt}
\setlength{\parskip}{5pt}

\newcommand{\mybox}[1]{\frame{\rule{0pt}{10pt}\ #1\ }}

\newcommand{\eps}{\varepsilon}
\newcommand{\ls}{\left\langle}
\newcommand{\rs}{\right\rangle}

\begin{document}

% Wooooow, si pa pridna :)
% Go Vesna! ^^
% Ah zdel se mi je, da se morm učit, pa me je preveč glava bolela, da bi dejansko kej pametnega razmislila. Tko sm bla pa uporabna :P

%%%%%%%%%%%%%%%%                    APROKSIMACIJA
% mislim, da so tu vse stvari z vaj. Edino kar manjka je Remezov postopek, ker se mi ga včerej ni dalo napisat.
\subsection*{Aproksimacija}
\textbf{Bernsteinovi polinomi:} \\
Bernsteinov polinom: $B_i^n(x) = \binom{n}{i} x^i (1-x)^{n-i}$, $B_i^n$ so baza prostora $\P_n$, $B_i^n \geq 0,
B_i^n(x) = B_{n-i}^n(1-x),
\sum_{i=0}^n B_i^n(x) = 1$,
$\sum_{i=0}^n \frac{i}{n} B_i^n(x) = x$,
$B_i^n(x) = \sum_{j=i}^n (-1)^{j-i} \binom{n}{j} \binom{j}{i} x^j$,
$B_i^n(x) = (1-x) B_i^{n-1}(x) + x B_{i-1}^{n-1}(x)$,
$(B_i^n)'(x) = n (B_{i-1}^{n-1}(x) - B_i^{n-1}(x))$,
$\int B_i^n(x) dx = \frac{1}{n+1} \sum_{j = i+1}^n B_j^{n+1}(x) + C$\\
Bernsteinov operator: $(B_nf)(x) = \sum_{i=0}^n f(\frac{i}{n}) B_i^n(x)$\\
$(B_nf)'(x) = n \sum_{i=0}^{n-1} \Delta f(\frac{i}{n}) B_i^{n-1}(x)$, kjer
$\Delta f(\frac{i}{n}) = f(\frac{i+1}{n}) - f(\frac{i}{n})$\\
$(B_nf)^{(k)}(x) = \frac{n!}{(n-k)!} \sum_{i=0}^{n-k} \Delta^kf(\frac{i}{n}) B_i^{n-k}(x)$,
kjer $\Delta^k f(\frac{i}{n}) = \Delta^{k-1} f(\frac{i+1}{n}) - \Delta^{k-1} f(\frac{i}{n})$\\
Za $f \colon [0,1] \to \R$ zvezno odvedljivo: $\lim_{n\to \infty} \|(B_nf)' - f'\|_{\infty} = 0$.\\
Weierstrassov izrek: Za $f \in C([a,b])$ velja $\text{dist}_{\infty} (f, \P_n) \to 0$.\\
Izrek: $\|f - B_n f\|_{\infty} \leq \omega(f, \frac{1}{\sqrt[4]{n}}) + \frac{\|f\|_{\infty}}{2 \sqrt{n}}$\\
% Kantorovičev polinom??\\ NE!
Ocena napake: $e_n = \|B_nf - f\|_{\infty} = C n^{-\alpha}$, $\frac{e_n}{e_m} =
(\frac{m}{n})^{\alpha}$, $\alpha =
\frac{\log(\frac{e_n}{e_m})}{\log(\frac{m}{n})}$.\\
Če je $f$ Lipschitzova s konstanto $c$, je $|f(x) - (B_nf)(x)| \leq  \frac{c}{2 \sqrt{n}}$\\
% Vesna, kaj je to? Rezultat ene naloge z vaj, ki sem ga pozabla v celoti prepisat :P
$(B_nf)(x) = \sum_{i=0}^{n+1} (f(\frac{i-1}{n}) \frac{i}{n+1} + f(\frac{i}{n}) \frac{n+1-i}{n+1}) B_i^{n+1}(x)$\\
Če je $f$ konveksna, je $B_nf \geq B_{n+1}f$.\\
Triki:\\
- 1 zapišeš kot vsoto Bern.\ polinomov\\
- C.-S.\ neenakost: $|\ls x,y\rs| \leq \|x\| \cdot \|y\|$

\textbf{Hiške:} $S$ so odsekoma linearne funkcije. Hiške so baza $S$. $\dim(S) = n+1$\\
$H_i(x) = \begin{cases}
\frac{x-x_{i-1}}{x_i - x_{i-1}}; & x \in [x_{i-1}, x_i]\\
\frac{x-x_{i+1}}{x_i - x_{i+1}}; & x \in [x_{i}, x_{i+1}]\\
0; & \text{ sicer}
\end{cases}$\\
Operator: $(I_1 f)(x) = \sum_{i = 0}^n f(x_i) H_i(x)$\\
Izrek: Če je $I_1$ linearen in pozitiven in če je $\lim_{\Delta x \to 0} \|I_1
f - f\|_{\infty} = 0$ za $f \in \P_i$ za $i = 0,1,2$, potem to velja za vsak $f
\in C([a,b])$. (Naš $I_1$ temu zadošča.)

\textbf{Element najboljše aproksimacije:} $X$ normiran prostor, $S \subset X$
podprostor, $f \in X$; iščemo $f^* \in S$: $\|f - f^*\| = \text{dist}(S,f) =
\inf_{s \in S} \{\|f-s\|\}$.\\
Izrek: Če imamo enakomerno konveksen Banachov prostor in zaprt podprostor, potem obstaja e.n.a.\\
Izrek: V končno razsežnih podprostorih e.n.a.\ vedno obstaja.\\
$X$ normiran vektorski prostor. $X$ je \textbf{enakomerno konveksen}, če
$\forall \eps > 0 \exists \delta >0: \forall x,y \in X: \|x\| = \|y\| = 1,
\text{ velja: } \|\frac{1}{2} (x+y)\| > 1-\delta \implies \|x-y\| < \eps$.\\
$X$ je \textbf{strogo normiran}, če za $x,y \in X$ za katera je $\|x+y\| =
\|x\| + \|y\|$, velja $x = \lambda y, \lambda \in \C$.\\
Množica $A$ je \textbf{strogo konveksna}, če za vse $a,b \in A, a \neq b$ in
vse $\alpha \in (0,1)$ velja $\alpha a + (1-\alpha) b \in \text{int} A$.\\
Krogla je \textbf{strogo konveksna} $\iff \forall x,y \in K(0,1), x \neq y,
\alpha \in (0,1)$, velja $\|\alpha x + (1-\alpha) y\| < 1$.\\
Izrek: $X$ z normo, v kateri so zaprte krogle strogo konveksne. Potem je $X$
strogo normiran.\\
Izrek: $X$ strogo normiran $\iff$ zaprta enotska krogla je strogo normirana.\\
Izrek: Množica e.n.a. je konveksna. \\
Izrek: $X$ Strogo normiran prostor. Potem za vsak $x\in X$ obstaja največ 1
e.n.a. (Unitarni prostori so strogo normirani).
\textbf{Remezov postopek:}
Za p.n.e.a.\ stopnje $n$ vzamemo množico $E$ z $n+2$ točkami, $E = \{x_0, \ldots, x_{n+1}\}$.
Potem rešimo sistem enačb, ki pravi da mora residual $r = f - p^\ast$ alternirati: $r(x_i) = (-1)^i m$.
Če zapišemo $p^* = \sum a_i x^i$, potem se sistem glasi:\\
$\begin{bmatrix}
  1 & 1 & x_0 & x_0^2 & \cdots & x_0^n \\
  -1 & 1 & x_1 & x_1^2 & \cdots & x_1^n \\
  \vdots & \vdots & \vdots & \vdots & \ddots & \vdots \\
  \pm 1 & 1 & x_{n+2} & x_{n+2}^2 & \cdots & x_{n+2}^n
\end{bmatrix}
\begin{bmatrix}
  m \\ a_0 \\  \vdots \\ a_n
\end{bmatrix}
=
\begin{bmatrix}
  f(x_0) \\ f(x_1) \\ \vdots \\ f(x_{n+2})
\end{bmatrix}
$ \\
Rešimo in zračunamo residual $r$, ki mora alternirajoče dosegati $m$. Pazi, da so koeficienti polinoma
po vrsti od $x^0$ do $x^n$!
(Ne nujno kot svojo normo, takrat je postopka konec, oz.~ko je $\|r\|_\infty - m < \eps$.)
Sedaj se ločita prvi in drugi Remezov postopek.
\textbf{Prvi:}
Izračunamo, kje je dosežen maksimum $|r|$ (ponavadi z odvajanjem),
označimo točko z $\xi$, $r(\xi) = \|r\|_\infty$. Sedaj točko $\xi$ vstavimo v $E$, tako da vržemo ven
eno izmed sosednjih, in sicer tisto, da $r$ še vedno alternira na $E$.
\textbf{Drugi:}
Ker $r$ alternira na $n+2$ točkah ima $n+1$ ničel $z_i$. Če dodamo še $z_{-1} = a, z_{n+1} = b$, potem
je na intervalih $[z_i, z_{i+1}]$ enako predznačen -- na vsakem intervalu najdemo ekstrem $y_i$. Teh je $n+2$ in
množico $\{y_i\}$ okličemo za nov $E$, starega pa pozabimo.

Včasih lahko dobimo p.n.e.a.\ brez Remezovega postopka:\\
- ga uganemo + Izrek (\emph{de La Vallée Poussin}): $f \in C([a,b]), p \in \P_n$, tako da $r = f-p$
alternirajoče doseže svojo normo v vsaj $n+2$ točkah. Tedaj je $p$ p.n.e.a.\ za $f$ na $[a,b]$.\\
- to na primer velja za polinome Čebiševa na $[-1, 1]$.

Remeza lahko delamo, če funkcije zadoščajo Harovemu pogoju.

\textbf{Polinomi Čebiševa} na $[-1,1]$: $T_n(x) = \cos(n \arccos (x)), x \in [-1,1]$,
$T_n(x) = 2x T_{n-1}(x) - T_{n-2}(x)$, \\
$T_0(x) = 1, T_1(x) = x, T_2(x) = 2x^2 - 1,
T_3(x) = 4 x^3 - 3x, T_4(x) = 8x^4 - 8x^2 + 1$.\\
$T_n(x)$ alternirajoče doseže $\pm 1$ v $n+1$ točkah in $\|T_n\|_{\infty} = 1$.\\
Med vsemi polinomi stopnje $\leq n$ in vodilnim koeficientom $1$, ima $2^{-n+1}
T_n$ najmanjšo $\infty$-normo na $[-1,1]$, hkrati pa izven $[-1,1]$ najhitreje
narašča.\\
Med vsemi polinomi $p \in \P_n$, za katere je $\|p\|_{\infty, [-1,1]} \leq 1$,
polinom $T_n$ največji vodilni koeficient.\\
Če je $p(x) = \sum_{i=0}^n a_i T_i(x)$, je $q(x) = p(x) - a_n T_n(x) =
\sum_{i=0}^{n-1} a_i T_i(x)$ p.n.e.a.\ za $p$ na $[-1,1]$ stopnje $n-1$.
$\|p-q\|_{\infty} = |a_n|$.

\textbf{Harov pogoj:} $\{f_i\}_{i=0}^n$ zvezne, zadoščajo Harovemu pogoju na
$[a,b]$, če je za vsake točke $a \leq x_0 < x_1 < \cdots < x_n \leq b: V(f;x) =
\det (f_j(x_i))_{i,j = 0}^{n} = \begin{vmatrix}
  f_0(x_0) & f_0(x_1) & \cdots & f_0(x_n) \\
  f_1(x_0) & f_1(x_1) & \cdots & f_1(x_n) \\
  \vdots & \vdots & \ddots & \vdots \\
  f_n(x_0) & f_n(x_1) & \cdots & f_n(x_n)
 \end{vmatrix} \neq 0.$ Če je $f_i(x) = x^i$, dobimo Vandermondovo determinanto.\\
 Zelo uporaben izrek: Harov pogoj je izpolnjen $\iff$ vsak ``polinom''
 $\sum_{i=0}^n \alpha_i f_i$ ima kvečjemu $n$ različnih ničel na $[a,b]$.
 Za take sisteme lahko uporabljamo Remezov postopek.

 \textbf{Metoda najmanjših kvadratov:} \\
 Izrek: $X$ evklidski prostor, $f \in X$, $f^* \in S \subset X$. $f^*$ je e.n.a.m.n.k.\ za $f$ v $S$
$\iff$ $f - f^* \bot S$.\\
 $(s_j)$ baza $S \subset X$, $f \in X$, $f^* \in S$, $f^* = \sum \alpha_j s_j$. Dobimo sistem $G
\alpha = b$, $G$ je Grammova matrika (simetrična in pozitivno definitna).\\
$ \begin{bmatrix}
  \ls s_1,s_1\rs  & \ls s_2, s_1\rs  & \cdots & \ls s_n, s_1\rs  \\
  \ls s_1,s_2\rs  & \ls s_2, s_2\rs  & \cdots & \ls s_n, s_2\rs  \\
  \vdots & \vdots & \ddots & \vdots \\
  \ls s_1,s_n\rs  & \ls s_2, s_n\rs  & \cdots & \ls s_n, s_n\rs
 \end{bmatrix}
 \begin{bmatrix}
  \alpha_1 \\ \alpha_2 \\ \vdots \\ \alpha_n
 \end{bmatrix}  = \begin{bmatrix}
 \ls f, s_1\rs  \\ \ls f, s_2\rs  \\ \vdots \\ \ls f, s_n\rs
 \end{bmatrix}$\\
 $G$ je nesingularna $\iff$ $s_1, \ldots, s_n$ so linearno neodvisne.\\
 Včasih gledamo vse zoženo le na nekaj točk $x = (x_i)_{i=1}^n$, tedaj je $G$ nesingularna $\iff$
imamo več kot $n+1$ različnih točk na $\P_n|_x$.\\
 \textbf{Ortonormirani sistemi polinomov:}
 Dan je nek skalarni produkt, želimo ortonormirano bazo polinomov stopnje $n$,
 $(Q_0, Q_1, \ldots, Q_n), \|Q_i\| = 1, \ls Q_i, Q_j\rs  = \delta_{ij}$.  \\
 Izberemo $Q_{-1} = 0, \tilde{Q}_0 = 1$. Računamo od $i=0$ dalje: \\
 $\beta_i = \|\tilde{Q}_i\| = \sqrt{\ls \tilde{Q}_i, \tilde{Q}_i \rs}$\\
 $Q_i = \tilde{Q}_i/\beta_i$ \\
 $\alpha_i = \ls xQ_i, Q_i\rs$, \\
 $\tilde{Q}_{i+1}(x) = (x - \alpha_i) Q_i(x) - \beta_i Q_{i-1}(x)$ in ponavljaš. \\
 Alternativno velja:
 $\beta_i = \ls x Q_i, Q_{i-1}\rs  = \|\tilde{Q}_i\|$ \\
Dobljeni $Q$-ji so ortonormirani, $\tilde{Q}_i$ pa ortogonalni.
Ker je Gramova matrika kar identiteta, zapišemo p.n.a.m.n.k. kar:
$p^* = \sum_{i=0}^n \ls f, Q_i \rs Q_i$.
% :)

%%%%%%%%%%%%%%%%                    INTERPOLACIJA
\subsection*{Interpolacija}
\textbf{Lagrangeeva} interpolacija: interpoliramo le vrednosti funkcije $f$ v različnih točkah.\\
\textbf{Hermitova} intrepolacija: interpoliramo vrednosti in vrednosti 1.\ odvodov funkcije $f$.

\textbf{Lagrangeev interpolacijski polinom:} $a \leq x_0 < x_1 < \cdots x_n \leq b$ delilne točke\\
$\displaystyle \ell_{n,j}(x)=\frac{\prod_{i=0,i\neq j}^n(x-x_i)}{\prod_{i=0,i\neq
j}^n(x_j-x_i)}$, $\ell_{n,j}(x_i) = \delta_{ij}$, Polinom, ki se z $f$ ujema v
$(x_i)_i$: $p(x) = \sum_{i=0}^nf(x_i)\ell_{n,i}(x)$.\\
$\{\ell_{n,j}\}_j$ so baza $\P_n$, $\sum_{j=0}^n \ell_{n,j} = 1$\\
Definiramo $\omega(x) = (x-x_0)\cdots(x-x_n)$. Velja $\ell_{n,i}(x) = \frac{\omega(x)}{(x-x_i)\omega'(x)}$.

Ocena napake: $f(x) = p(x) + \frac{f^{(n+1)}(\xi)}{(n+1)!}\omega(x)$, kjer je
$\xi$ nekje na intervalu, ki ga določajo $x_i$ in $x$. To se prevede na:
$|f(x) - p(x)|  = |\omega(x) [x_0, \ldots, x_n, x]f| =
|\omega(x) \frac{f^{(n+1)}(\xi)}{(n+1)!}| \leq \|\omega\|_\infty
\frac{\|f^{(n+1)}\|_\infty}{(n+1)!}$\\
V več dimenzijah: kolokacijska matrika $[\lambda_j(s_i)]_{i,j = 0}^n$,
$\lambda_j$ evaluacije v točkah $x_j$, $s_i$ bazne funkcije prostora.

Izračun vrednosti polinoma, ki je dan z $d_i = [x_0, \ldots, x_i]f$:\vspace{-\baselineskip}
\small
\begin{verbatim}
v = d_n
for i = n-1:-1:0
    v = d_i + (x - x_i) v
\end{verbatim}
\normalsize

\textbf{Deljene diference in Newtonova oblika interpolacijskega polinoma:}\\
$[x_0, \ldots, x_k]f$ je koeficient pred $x^k$ v polinomu stopnje $\leq k$,
ki se z $f$ ujema v teh $k+1$ točkah.\\
% no ja, ni ravno vodilni koeficient, je koeficient pri x^k, kar je lahko tudi 0, ker je polinom lahko manjši...
% ampak ja.
%nene, maš prav, zato je šlo men tole razmišljanje tolk počas, k mam napačno def. v glavi
Če so točke paroma različne: $[x_i]f = f(x_i)$, ostalo izračunamo po
rekurzivni formuli: \\$[x_0, \ldots, x_k]f = \frac{[x_0, \ldots, x_{k-1}]f -[x_1, \ldots, x_k]f }{x_0-x_k}$. Če
so točke $x_0$ do $x_k$ enake, je $[x_0, \ldots, x_k]f = \frac{f^{(k)}(x_0)}{k!}$.\\
Zapis interpolacijskega polinoma: \\
$p(x) = f[x_0] + f[x_0, x_1] (x-x_0) +  f[x_0, \dots, x_n](x-x_0)\cdots (x-x_{n-1}) =
\sum_{j=0}^n \prod_{i=0}^{j-1}(x-x_i) [x_0, \ldots, x_j]f$\\
Izrek (ocena napake): $f \in C^{n+1}([a,b])$, $f(x) - p(x) = \omega(x) [x_0, x_1, \ldots x_n, x]f$,
pri čemer obstaja $\xi \in (\min x_i, \max x_i)$, da je
$[x_0, x_1, \ldots x_n, x]f = \frac{f^{(n+1)} (\xi)}{(n+1)!}$, torej kot pri Lagrangu (duh, sej je isti).\\
Izrek: $x_0, \ldots, x_n$ ne nujno urejene po velikosti, $f \in
C^{n+1}([a,b])$, $h = \max_i |x_{i+1} - x_i|$, $I = [\min_i x_i, \max_i x_i]$.
Velja: $\|f-p\|_{\infty, I} \leq \|f^{(n+1)}\|_{\infty, I} \frac{1}{4 (n+1)}
h^{n+1}$.\\
\textbf{Posplošeni Hornerjev algoritem:} \vspace{-\baselineskip}
\small
\begin{verbatim}
a_n = [x_0, ..., x_n]f
b_n = a_n
for i = (n-1):0
    b_i = a_i + b_(i+1) (x-x_i)
return b_0
\end{verbatim}
\normalsize
Velja: $b_0 = p(x) = [x]p$ in $b_i = [x_0, \ldots, x_{i-1}, x]p$, $b_n$ so
koeficienti razvoja polinoma $p$ po bazi na točkah $x_0, \ldots, x_{n-1}, x$.\\
S tem lahko dobimo Taylorjev razvoj okoli točke $a$ -- postopek ponavljamo na
dobljenih koeficientih, dokler v bazo ne vrinemo samo $x-a$\\
Če so $x_0, \ldots, x_n$ paroma različne: $[x_0, \ldots, x_n]f = \sum_{i=0}^n
\frac{f(x_i)}{\omega'(x_i)}$ in $\sum_{i=0}^n \frac{x_i f(x_i}{\omega'(x_i)} =
\frac{x_n [x_1, \ldots, x_n]f - x_0 [x_0 \ldots, x_{n-1}]f}{x_n - x_0}$\\
Leibnitzova formula: $[x_0, \ldots, x_n](g \cdot h) = \sum_{i=0}^n [x_0,
\ldots, x_i]g \cdot [x_i, \ldots, x_n]h$.\\
\textbf{1.\ Newtonova oblika: }$x_i = x_0 + ih$, $f_i = f(x_i)$, $\Delta^0 f_i
= f_i$, $\Delta^r f_i = \Delta^{r-1} f_{i+1} - \Delta^{r-1} f_i$, polinom $p(x)
= p(x_0 + th) = \sum_{i=0}^n \binom{t}{i} \Delta^i f_0$, $t = (x-x_0)/h$.\\
Velja: $[x_0, \ldots, x_n]f = \Delta^n f_0 \frac{1}{n! \cdot h^n}$\\
\textbf{Lebesgueova neenakost:} $X$ normiran vektorski prostor, $\|\cdot\|$
norma, $S \subset X$ podprostor. Naj bo $P \colon X \to S$ linearni projektor.
Tedaj velja  $\forall f \in X\colon \|Pf\| - \|f\| \leq \|f - Pf\| \leq (1 +
\|P\|) \text{ dist} (f, S)$.\\

%%%%%%%%%%%%%%%%                    ZLEPKI
\subsection*{Zlepki}
Standardna metoda pri dokazih: zlepek obravnavaš na vsakem intervalu posebej!\\
\textbf{Odsekoma linearni: } $S_{1, x}$\\
\textbf{$I_1$: }za $x \in [x_i, x_{i+1}]$ je $(I_1f)(x) = [x_i]f + (x-x_i)
[x_i, x_{i+1}]f = f(x_i) + (x-x_i) \frac{f(x_i) - f(x_{i+1})}{x_i - x_{i+1}}$\\
Izrek: $\|f - I_1f\|_{\infty, [a,b]} \leq \frac{1}{8} \Delta x^2
\|f^{(2)}\|_{\infty, [a,b]}$, kjer $\Delta x = \max_i \Delta x_i$\\
Izrek:  $\|f - I_1f\|_{\infty, [a,b]} \leq 2 \text{dist}(f, S_{1,x})$\\
Izrek:  $\|f - I_1f\|_{\infty, [a,b]} \leq \omega(f, \Delta x)$, kjer je
$\omega(f, h) = \max  \{|f(x) - f(y)|\; ; \; x,y \in [a,b], |x-y| < h\} $\\
\textbf{$L_1$: }(interpolira po metodi najmanjših kvadratov) $L_1 f =
\sum_{i=0}^n \alpha_i H_i$, rešimo normalni sistem ($G \alpha = b, G = [\ls
H_i, H_j\rs]_{i,j= 0}^n, b = [\ls f, H_j \rs]_{j=0}^n$, $G$ je tridiagonalna,
strogo diagonalno dominantna)

\textbf{Odsekoma parabolični:}\\
Na vsakem intervalu interpoliramo v točkah $x_i, \frac{x_i + x_{i+1}}{2},
x_{i+1}$ z vrednostmi $f(x_i), v_i, f(x_{i+1})$. $v_i$ izračunamo iz sistema
enačb ($z_i'(x_i) = z_{i-1}'(x_i)$) $\frac{4 v_{i+1}}{\Delta x_{i+1}} + \frac{4
v_i}{\Delta x_i} = \frac{f(x_i)}{\Delta x_i} + \frac{f(x_{i+2)}}{\Delta
x_{i+1}} + 3 f(x_{i+1})(\frac{1}{\Delta x_i} + \frac{1}{\Delta x_{i+1}})$ za
$i=0,1,\ldots,n-2$, določimo $v_{n-1}$ ter vstavimo v $z_i(x) = f(x_i) +
(x-x_i)\frac{f(x_{i+1} - f(x_i)}{\Delta x_i} + (x-x_i)(x-x_{i+1})
\frac{2(f(x_i) + f(x_{i+1} -2v_i)}{(\Delta x_i)^2}$

\textbf{Odsekoma kubični:}\\
\textbf{Hermitov kubični zlepek:} Če poznamo funkcijo in prvi odvod v
$(x_i)_{i=0}^n$: interpoliramo na vsakem odseku posebej v
točkah $(x_i, x_i, x_{i+1}, x_{i+1})$.\\
$z_i(x) = f(x_i) + (x-x_i) f'(x_i) + (x-x_i)^2 \frac{[x_i, x_{i+1}]f -
f'(x_i)}{x_{i+1} - x_i} + (x-x_i)^2 (x-x_{i+1}) \frac{f'(x_{i+1}) - 2 [x_i,
x_{i+1}]f - f'(x_i)}{(x_{i+1} - x_i)^2}$\\
Napako ocenimo enako kot pri deljenih diferencah.\\
\textbf{Poln kubični zlepek:} Če poznamo le vrednosti funkcije, zahtevamo 2x
zvezno odvedljivost zlepka. Uvedemo parametre $s_i = z'(x_i)$. \\
Rešimo sistem enačb $\frac{s_{i-1}}{\Delta x_{i-1}} + 2 s_i
\left(\frac{1}{\Delta x_{i-1}} + \frac{1}{\Delta x_i}\right) +
\frac{s_{i+1}}{\Delta x_i} = 3\left(\frac{[x_i, x_{i+1}]f}{\Delta x_i} +
\frac{[x_{i-1}, x_i]f}{\Delta x_{i-1}}\right)$ za $i=1,2,\ldots,n-1$, izberemo
$s_0$ in $s_n$ in vstavimo v $z_i(x) = f(x_i) + (x-x_i)s_i +
(x-x_i)^2\frac{[x_i, x_{i+1}]f - s_i}{\Delta x_i} +
(x-x_i)^2(x-x_{i+1})\frac{s_{i+1} + s_i - 2[x_i, x_{i+1}]f}{\Delta x_i^2}$. Za
poln zlepek vzamemo $s_0 = f'(x_0), s_n = f'(x_n)$.

\textbf{B-zlepki:} Baza prostora zlepkov. So nenegativni, z lokalnim nosilcem
in tvorijo particijo enote.\\
$z = \sum_i \alpha_i B_{i,k}$, $\text{supp} B_{i,k} \subseteq [t_i, t_{i+k+1}]$\\
Vozli: $t_i \leq t_{i+1} \leq \ldots \leq t_{i+k+1}; \; t_i < t_{i+k+1}$\\
Če se vozel ponovi $j$-krat, v tej točki zahtevao zveznost odvoda reda $k-j$.\\
$B_{i,k} (x) = (t_{i+k+1} - t_i) [t_i, \ldots, t_{i+k+1}](\cdot - x)_+^k$, kjer
je $(\cdot - x)_+^k = \max\{0, (\cdot - x)^k\}$\\ Rekurzija: $B_{i,k}(x) =
\frac{x-t_i}{t_{i+k}-t_i} B_{i, k-1} (x) + \frac{t_{i+k+1} - x}{t_{i+k+1} -
t_{i+1}} B_{i+1, k-1} (x)$\\ $B_{i,0} (x) = 1 \cdot \mathbb{1}(x \in [t_i,
t_{i+1})$\\ \textbf{De Bohrov algoritem za računanje vrednosti:}
$\alpha_i^{[r]} (x) =
\begin{cases}
\alpha_i ;& r = 0\\
\frac{(x-t_i) \alpha_i^{[r-1]} (x) + (t_{i+k+1-r} - x) \alpha_{i-1}^{[r-1]}}{t_{i+k+1} - t_i};& r > 0
\end{cases}$\\
Napišeš si tabelo, odgovor je potem $\alpha_j^{[k]}(x)$, kjer $t_j \leq x <
t_{j+1}$. Tabelo delaš le po tistih $i$, kjer je $B_{i,k} \neq 0$. %???

\hspace*{\fill} Avtorji: Vesna Iršič, Jure Slak, generacija 2015/16
\end{document}
