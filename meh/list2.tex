\documentclass[a4paper,10pt]{article}
\usepackage[slovene]{babel}
\usepackage[utf8]{inputenc}
\usepackage[T1]{fontenc}
\usepackage{lmodern}
\usepackage{url}
\usepackage{graphicx}
\usepackage[usenames]{color}
\usepackage[reqno]{amsmath}
\usepackage{amssymb,amsthm}
\usepackage{enumerate}
\usepackage{array}
\usepackage[bookmarks, colorlinks=true, %
linkcolor=black, anchorcolor=black, citecolor=black, filecolor=black,%
menucolor=black, runcolor=black, urlcolor=black, pdfencoding=unicode%
]{hyperref}
\usepackage[
  paper=a4paper,
  top=1.5cm,
  bottom=1.5cm,
%    textheight=24cm,
  textwidth=18cm,
  ]{geometry}

\usepackage{icomma}
\usepackage{units}

\newtheorem{izrek}{Izrek}
\newtheorem{posledica}{Posledica}

\theoremstyle{definition}
\newtheorem{definicija}{Definicija}
\newtheorem{opomba}{Opomba}
\newtheorem{zgled}{Zgled}

\def\R{\mathbb{R}}
\def\N{\mathbb{N}}
\def\Z{\mathbb{Z}}
\def\C{\mathbb{C}}
\def\Q{\mathbb{Q}}

\newenvironment{itemize*}%
{
\vspace{-6pt}
\begin{itemize}
\setlength{\itemsep}{0pt}
\setlength{\parskip}{2pt}
}
{\end{itemize}}

\newenvironment{enumerate*}%
{
\vspace{-6pt}
\begin{enumerate}[(1)]
\setlength{\itemsep}{0pt}
\setlength{\parskip}{2pt}
}
{\end{enumerate}}

\newcommand{\mytitle}{Mehanika}
\title{\mytitle}
\author{Jure Slak}
\date{\today}
\hypersetup{pdftitle={\mytitle}}
\hypersetup{pdfauthor={Jure Slak}}
\hypersetup{pdfsubject={}}

\pagestyle{empty}

\setlength{\parindent}{0pt}

\newcommand{\dx}{\ensuremath{\,\mathrm{d}x}}
\newcommand{\dt}{\ensuremath{\,\mathrm{d}t}}

\DeclareMathOperator{\grad}{grad}

\let\theta\vartheta

% vektorji
\newcommand{\vzeta}{\vec{\zeta}}
\newcommand{\dzeta}{\dot{\vzeta}}
\newcommand{\ddzeta}{\ddot{\vzeta}}
\newcommand{\vomega}{\vec{\omega}}
\newcommand{\domega}{\dot{\vomega}}
\newcommand{\va}{\vec{a}}
\newcommand{\vr}{\vec{r}}
\newcommand{\dr}{\dot{\vr}}
\newcommand{\ddr}{\ddot{\vr}}
\newcommand{\er}{\vec{e}_r}
\newcommand{\eq}{\vec{e}_{\theta}}
\newcommand{\Pt}{P_{\ast}}

\begin{document}

\subsubsection*{Premočrtno gibanje}
Osnovni vektorji: $\vr = r\er, \qquad \dr = \dot{r}\er + r\dot{\theta}\eq,
\qquad \ddr = (\ddot{r} - r\dot{\theta}^2)\er + (2\dot{r}\dot{\theta} +
r\ddot{\theta})\eq$

Newtonov zakon na krivulji: $m(\ddot{s}\vec{e}_T + \kappa\dot{s}^2\vec{e}_N)
= \vec{F} + \vec{S}$

Sila je potencialna, če je $\vec{F} = -\grad U = - \frac{\partial F}{\partial
r}\er$. \\
Energijska enačba: $\frac12m\dot{x}^2 + U(x) = E_0$ \\
Ravnovesna lega: $U'(x) = 0$, stabilna če $U''(x) > 0$. \\

\subsubsection*{Relativno gibanje}
Količine s $'$ so zapisane v AKS. \\
$P' = P_0' + Q(t)(P-P_0)$ \\
$W = Q^T\dot{Q}, \qquad  W\va = \vomega \times \va$ \\
$\vzeta = P - P_0, \qquad \vec{v}_{rel} = \dzeta, \qquad \vec{a}_{rel} =
\ddzeta$
\\
Newtonove enačbe v RKS: \\[6pt]
$m\ddzeta = \vec{F} -
\underbrace{m\va_0}_{\eqref{sez:in}} -
\underbrace{m\vomega \times (\vomega \times \vzeta)}_{\eqref{sez:cen}} -
\underbrace{m\domega\times\vzeta}_{\eqref{sez:inrot}} -
\underbrace{2m\vomega\times\dzeta}_{\eqref{sez:cor}}$

\begin{enumerate*}
  \item inercijska sila, $a_0$ je pospešek RKS glede na AKS,  \label{sez:in}
  \item centrufugalna sila,                                   \label{sez:cen}
  \item inercijska sila zaradi kotnega pospeška,              \label{sez:inrot}
  \item Coriolisova sila.                                     \label{sez:cor}
\end{enumerate*}

\subsubsection*{Sistem materialnih točk in togo telo}
Masno središče za Sistem točk $P_i$ z masami $m_i$ in homogeno togo telo: \\
$ \displaystyle \Pt = O + \frac1m \sum_{i=1}^n (P_i - O) m_i \qquad \Pt =
\frac1V \int_B (x\vec{\imath} + y\vec{\jmath} + z\vec{k})dV $

Aksiomi za togo telo: \\
$m\ddot{\Pt}' = \vec{F}'$ \\
$\displaystyle \frac{d\vec{L}'(O')}{dt} = \vec{N}'(O')$ \\
$\vec{N}'(O') = \vec{N}(P_0') + (P_0' - O') \times \vec{F}'$

Za kinetično energijo velja:
$T = T_\ast + T_r$, kjer je $T_\ast = \frac12 m\vec{v}_\ast$ in $T_r = \frac12 \vomega \cdot J \vomega$

Velja energijski zakon: $T + U = E_0$.

\subsubsection*{Vztrajnostni tenzor}
V telesni bazi je vztrajnostni tenzor enak:
\[ J = \int
  \begin{bmatrix}
    y^2 + z^2 & -xy & -xz \\
    -xy & x^2 + z^2 & -yz \\
    -xz & -yz & x^2 + y^2
\end{bmatrix} dm \]

Znani vztrajnostni tenzorji v telesni bazi: \\
krogla: $J = \frac25mr^2I$, sfera: $J = \frac23mr^2I$, palica okrog sredine:
$J = \frac{1}{12}ml^2$, okrog krajišča: $J = \frac{1}{3}ml^2$, \\disk okrog
središča: $J = \frac12mr^2$, okrog premera: $J = \frac14mr^2$, obroč: $J =
mr^2$, \\
elipsoid: $J =
  \begin{bmatrix}
    \frac15m(b^2+c^2) & & \\ & \frac15(a^2+c^2) & \\ & & \frac15m(a^2+b^2)
  \end{bmatrix}$ \\
stožec ($x^2 + y^2 = z^2)$ z radijem $r$ in višino $h$ okrog $z$ osi:
  $\begin{bmatrix}
    \frac35 mh^2+\frac{3}{20}mr^2 & & \\ & \frac35mh^2 + \frac{3}{20}mr^2 & \\
    & & \frac{3}{10}mr^2
  \end{bmatrix}$ \\
pokončen valj okrog $z$ osi:
  $J = \begin{bmatrix}
    \frac{1}{12}m(3r^2+h^2) & & \\ & \frac{1}{12}m(3r^2 + h^2) & \\ & &
    \frac12 mr^2
  \end{bmatrix}$

V prostorsko bazo ga pretvorimo: $J' = Q^TJQ$. \\
Steinerjev izrek: $J(P_0) = J(\Pt) + m|\Pt - P_0|^2I - m(\Pt - P_0) \otimes (\Pt
- P_0)$

\subsubsection*{Rotacije}
Rotacije okrog spremenljive osi:
$R(\vec{e}(t),\varphi(t))\vec{r} = \cos\varphi\vec{r} +
(\vec{e}\vec{r})(1-\cos\varphi)\vec{e} + \sin\varphi(\vec{e} \times \vec{r})$ \\
Rotacijska matrika za rotacijo: \\
\[
  R(\vec{\imath}, \varphi) =
  \begin{bmatrix}
    1 & 0 & 0 \\
    0 & \cos\varphi & -\sin\varphi \\
    0 & \sin\varphi & \cos\varphi
  \end{bmatrix} \quad
  R(\vec{\jmath}, \varphi) =
  \begin{bmatrix}
    \cos\varphi & 0 & \sin\varphi  \\
    0           & 1 & 0   \\
    -\sin\varphi & 0 & \cos\varphi
  \end{bmatrix} \quad
  R(\vec{k}, \varphi) =
  \begin{bmatrix}
    \cos\varphi & -\sin\varphi & 0 \\
    \sin\varphi & \cos\varphi & 0 \\
    0 & 0 & 1
  \end{bmatrix}
\]

Če je dana rotacijska matrika, potem velja $1 + 2\cos\varphi =
\operatorname{sl}(Q)$. Vektor rotacije je lastni vektor, smer pa določimo tako,
da en vektor preslikamo.

\subsubsection*{Eulerjeve dinamične enačbe}
Vektorska oblika: $J(\Pt)\domega + \vomega \times J(\Pt)\vomega =
\vec{N}(\Pt)$
\begin{align*}
  J_1\dot{\omega}_1 - \omega_2\omega_3(J_2 - J_3) &= N_1 \\
  J_2\dot{\omega}_2 - \omega_3\omega_1(J_3 - J_1) &= N_2 \\
  J_3\dot{\omega}_3 - \omega_1\omega_2(J_1 - J_2) &= N_3
\end{align*}

\end{document}
% vim: syntax=tex
% vim: spell spelllang=sl
% vim: foldlevel=99
